% !TEX root = ../zenkoku.tex
\section{結果}
集計方法

今回は熟練した開発者を1人と想定し,この開発者への依存度が高いようなシステムを洗い出すことにした.熟練者を1人想定し,理解度の数値を `熟練者 - 熟練者を除いた平均` によって求める.この数値を属人性の高さとする.

結果を表hogeに示す

表1.属人性の高さ(3pt以上)
memo: 右側の個別の開発者のやつ消す
memo: 認識で0を選択した人は省く,開発に関わっていない人も含まれてしまう

属人性が高いものがn個
低いものhoge個

また,ドキュメントの充実度についての設問における熟練者を除いた平均をドキュメントの充実度とした.

表2.ドキュメントの充実度(2ptより小さい)

今回,属人性が高いがドキュメントの充実度が低いものについて以下のように可視化された.

表3.属人性が高いがドキュメントの充実度が低いもの(TRUEのみ)

・アンケートによって執筆すべきとされるシステムを洗い出すことが出来た
・ただブルーム・タキソノミーの6分類を学習者に評価してもらうことによる同じ数値でも認識の相違がある
