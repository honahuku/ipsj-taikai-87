% !TEX root = ../zenkoku.tex
\section{結果}
アンケートは部の開発者全員である9人全員から回答を得た.

集計方法としては熟練した開発者を1人想定し,この開発者への依存度が高いようなシステムを洗い出すことにした.
理解度は値が高いほど理解が深いものとし, `熟練者の理解度 - 熟練者を除いた平均` によって求めた数値を属人性の高さとする.

結果を次の表\ref{tab:systems}に示す\newline

\scriptsize
\begin{tabular}{|p{2.8cm}|p{2.2cm}|p{1.4cm}|} % 列幅を指定
    \hline
    \textbf{サービス名} & \textbf{理解度の平均(熟練者を除く)} & \textbf{属人性の高さ} \\
    \hline
    システムK & 0.7 & 3.3 \\
    \hline
    システムA & 1.6 & 4.4 \\
    \hline
    システムH (frontend) & 1.6 & 4.4 \\
    \hline
    システムB (frontend) & 1.7 & 4.3 \\
    \hline
    システムN (frontend) & 1.9 & 4.1 \\
    \hline
    システムN (backend) & 1.9 & 4.1 \\
    \hline
    システムE & 2.0 & 0.0 \\
    \hline
    システムH (backend) & 2.0 & 4.0 \\
    \hline
    システムB (backend) & 2.3 & 3.7 \\
    \hline
    ツール1 & 2.4 & 2.6 \\
    \hline
    システムF (frontend) & 2.6 & 1.4 \\
    \hline
    システムF (backend) & 2.7 & 1.3 \\
    \hline
    インフラに関連する設問1 & 2.9 & 3.1 \\
    \hline
    インフラに関連する設問2 & 2.9 & 0.1 \\
    \hline
    システムD & 3.0 & 3.0 \\
    \hline
    システムM & 3.0 & 3.0 \\
    \hline
    ツール3 & 3.3 & 1.7 \\
    \hline
    ツール5 & 3.4 & 1.6 \\
    \hline
    ツール6 & 3.6 & 1.4 \\
    \hline
    システムO & 3.7 & 0.3 \\
    \hline
    システムJ & 3.7 & 1.3 \\
    \hline
    ツール4 & 3.7 & 1.3 \\
    \hline
\end{tabular}
\vspace{0.1cm} % 表とキャプションの間の余白
\captionof{table}{システムに対する理解度の平均と属人性の高さ}
\label{tab:systems}
\normalsize % 他の部分の文字サイズを元に戻す
\vspace{0.3cm}


属人性の高さとして 3pt を閾値としたときに

属人性が高いものがn個
低いものhoge個

% ここヒストグラム書いても良いかも.線の値が大きくなるほど属人性が高いもの

% memo: 右側の個別の開発者のやつ消す

% memo: 認識で0を選択した人は省く,開発に関わっていない人も含まれてしまう

