% !TEX root = ../zenkoku.tex
\section{はじめに}
ソフトウェア開発において,チーム内で熟練した開発者は知識の共有を目的として社内の他の開発者向けに対し開発・運用しているシステムのドキュメントを作成することがある.
例としては既存システムを開発者感で引き継ぐ際に知識の断絶を防ぐために作成されるもの,組織に新しく参加した開発者に知識を共有するためのもの,個人の属人性が高い知識・運用手順をチーム全体に共有し残すためのもの等が挙げられる.

しかし,熟練者が執筆したドキュメントは他の開発者にとって十分な情報を満たしたものとならないことがある.
そこで,ドキュメントが開発者にとってシステムを理解することの手助けになるかを計測するために理解を計測することにする.
ただ理解度は個人の認知に基づく指標であるため,本稿では教育心理学にて知られている改訂版ブルーム・タキソノミー\cite{bib:nakao}という枠組を使い,バイアスを極力取り除いた(TODO: ここ言い換えがほしい)理解度の数値化を行う.

\section{理解度を計測する指標}
理解度を個人の認知だけではなく,指標を元に多角的に評価するための枠組みは複数あるが,そのうちの1つである改訂版ブルーム・タキソノミーに注目することにする.

ブルーム・タキソノミーでは,学習者の行動を認知的領域,情意的領域,精神運動的領域の3つに分類する.


ブルーム・タキソノミーは土木分野においても技術力を整理するものとして用いられた事例がある.\cite{bib:miyahara}
本稿では教育学にて提唱されている「改訂版ブルーム・タキソノミー」における認知的領域に注目し,記憶,理解,応用,分析,評価,創造,の6段階に基づいたアンケート作成する.
作成したアンケートをドキュメント利用者に回答してもらい認識を測定する.
測定結果をドキュメント作成者へフィードバックし,熟練者と他の開発者の間に生じる認識齟齬を改善することを目指す.

本稿ではシステムに対する理解度(自己認識)をアンケートによって回答してもらい,その結果を元にドキュメンテーションに役立てる

% TODO: その結果を元にドキュメンテーションに役立てる,部分について説明が足りていなさそう

% 以下の内容を追記したいかも
% TODO: 誰も知らないがドキュメントも無いものについては今回は対象外
% ドキュメントは書いてもきりが無いので,優先度を付けて対応したい(背景)
% ドキュメントを書くことにより開発する時間が減る(追記したい),が書かなければいけない(こっちは述べてる)

% TODO: 属人性が高い(だれか詳しい人はいる)が,ドキュメントが無いものを洗い出し,その詳しい人にドキュメントを書いてもらうことを目的としていた

% TODO: ドキュメント執筆という方法で記録を残すことにした理由(ペアプロとか他の知見共有方法はある)をもっと書いても良いかも
