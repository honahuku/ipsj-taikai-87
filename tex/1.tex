% !TEX root = ../zenkoku.tex
\section{はじめに}
ソフトウェア開発において,チーム内で熟練した開発者は知識の共有を目的として社内の他の開発者向けに対し開発・運用しているシステムのドキュメントを作成することがある.例としては既存システムを開発者感で引き継ぐ際に知識の断絶を防ぐために作成されるもの,組織に新しく参加した開発者に知識を共有するためのもの,個人の属人性が高い知識・運用手順をチーム全体に共有し残すためのもの等が挙げられる.

しかし,熟練者が執筆したドキュメントは他の開発者にとって十分な情報を満たしたものとならないことが多い.これは熟練者がすでに理解している事象に対し,理解が追いついていない他の開発者の理解度を認識することが難しいためである.この熟練者と他の開発者の間に生じる認識の差を改善するために,開発者の認識や理解度を多角的・体系的に評価する枠組みが必要となる.この枠組みとしては改訂版ブルーム・タキソノミーが教育心理学にて広く知られている\cite{bib:nakao}.

ブルーム・タキソノミーでは,学習者の行動を認知的領域,情意的領域,精神運動的領域の3つに分類したものである.ブルーム・タキソノミーは土木分野においても技術力を整理するものとして用いられた事例がある.\cite{bib:miyahara}

本稿では教育学にて提唱されている「改訂版ブルーム・タキソノミー」における認知的領域に注目し,記憶,理解,応用,分析,評価,創造,の6段階に基づいたアンケート作成する.作成したアンケートをドキュメント利用者に回答してもらい認識を測定する.測定結果をドキュメント作成者へフィードバックし,熟練者と他の開発者の間に生じる認識齟齬を改善することを目指す.
