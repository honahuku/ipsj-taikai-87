% !TEX root = ../zenkoku.tex
\section{はじめに}
ソフトウェア開発において,チーム内で熟練した開発者(以降,熟練者)はドキュメントを作成することがある.\cite{bib:ozawa}

% FIXME: なんか構造がおかしい,どのドキュメントに焦点を当てるかはもっと簡潔にかつ後に触れたほうが良さそう
ソフトウェア開発におけるドキュメントには,会社等の組織の中で開発されるシステムに対するドキュメント,OSS開発プロジェクトで開発されるシステムのドキュメント,ある組織が別の組織・個人に対して公開する開発者向けSDKに対するドキュメントなど複数の状況・用途が考えられるが,今回は山川が所属する株式会社の事業部の開発組織で開発・運用している広告配信システムの各サブシステム\footnote{全社共通で使われるようなシステムやツールではない}を対象に,熟練者が社内の他の開発者向に対し知識の共有を目的として作成するドキュメントに焦点を当てる.

% FIXME: もう少し丁寧な導入無いかな
しかし,熟練者が執筆したドキュメントは他の開発者にとって十分な情報を満たしたものとならないことがある.
教育現場においては教授者の知識の状態が教授学習法の判断に与える影響が実証されている\cite{bib:kaneda}.
特に,熟練者(教授者)が自身の持つ課題知識(問題解法や正解に関する知識)のみを基に判断を行った場合、学習者の立場や知識状態を正確に把握できない「知識の呪縛」が発生することが示されている.
情報工学においても,熟練者が自身の持つ知識を前提に記述を行う一方で,学習者の知識状態や具体的なニーズを正確に考慮できない場合が考えられる.

そこで,ドキュメントが開発者にとってシステムを理解することの手助けになるかを計測するために理解度を計測することにする.
ただ理解度は個人の認知に基づく指標であるため,本稿では教育心理学にて知られている改訂版ブルーム・タキソノミー\cite{bib:nakao}という枠組を使い,バイアスを極力取り除いた(TODO: ここ言い換えがほしい)理解度の数値化を行う.

\section{理解度を計測する指標}
理解度を個人の認知だけではなく,指標を元に多角的に評価するための枠組みは複数あるが,そのうちの1つである改訂版ブルーム・タキソノミーに注目することにする.

ブルーム・タキソノミーでは,学習者の行動を認知的領域,情意的領域,精神運動的領域の3つに分類する.


ブルーム・タキソノミーは土木分野においても技術力を整理するものとして用いられた事例がある.\cite{bib:miyahara}
本稿では教育学にて提唱されている「改訂版ブルーム・タキソノミー」における認知的領域て提唱されている記憶,理解,応用,分析,評価,創造,の6段階を情報工学に適用し,これに基づいたアンケート作成する.
作成したアンケートをドキュメント利用者に回答してもらい認識を測定する.
測定結果をドキュメント作成者へフィードバックし,熟練者と他の開発者の間に生じる認識齟齬を改善することを目指す.

本稿ではシステムに対する理解度(自己認識)をアンケートによって回答してもらい,その結果を元にドキュメンテーションに役立てる

% TODO: その結果を元にドキュメンテーションに役立てる,部分について説明が足りていなさそう

% 以下の内容を追記したいかも
% TODO: 誰も知らないがドキュメントも無いものについては今回は対象外
% ドキュメントは書いてもきりが無いので,優先度を付けて対応したい(背景)
% ドキュメントを書くことにより開発する時間が減る(追記したい),が書かなければいけない(こっちは述べてる)

% TODO: 属人性が高い(だれか詳しい人はいる)が,ドキュメントが無いものを洗い出し,その詳しい人にドキュメントを書いてもらうことを目的としていた

% TODO: ドキュメント執筆という方法で記録を残すことにした理由(ペアプロとか他の知見共有方法はある)をもっと書いても良いかも
