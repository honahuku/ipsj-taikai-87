% !TEX root = ../zenkoku.tex
\section{はじめに}
% 情報工学におけるドキュメントの有用性
ソフトウェア開発において,チーム内で熟練した開発者(以下,熟練者)は知識の共有を目的として社内の他の開発者向けに対し開発・運用しているシステムのドキュメントを作成することがある.
情報工学においては,メンバーの異動・退職などの引き継ぎ,ドキュメントによる開発者の知識・開発速度上昇などを目的としてドキュメントの作成(以下,ドキュメンテーション)とドキュメントが重要なものであると位置づけられている\cite{bib:ozawa}.

% ドキュメンテーションの課題
しかし,ドキュメンテーションによりそれを担当した開発者が実装等に使えた時間は減ることになってしまう.
またドキュメントは充足したという状況を計測することが難しく,ドキュメンテーション対象のシステム数が過剰である,ドキュメンテーション対象に優先順位を付けることが難しい,などの問題も存在する.

% 課題詳細・先行研究
さらに,ドキュメント執筆者はドキュメントの対象者の理解度がわからない,という問題も存在する.
この現象は教育学における,すでに理解した情報を知らないもの想定することは難しいこと\footnote{「知識の呪縛」と呼ばれる}\cite{bib:kaneda}と類似している.
また,改訂版ブルーム・タキソノミー(以下,ブルーム・タキソノミー)という概念では認知的領域を6段階に分類し,学習対象についての理解度を計測することが出来る.

% 本稿でどう解決していきたいと考えているか
理解度を計測することが可能であるならば,理解度を元にしたドキュメンテーション対象の絞り込みや優先度設定,ドキュメントによる理解度向上の目標設定なども可能となる.
本稿では,ブルーム・タキソノミーを元に作成したアンケートの実施によってシステムに対しての理解度を計測し,これを元にドキュメンテーションを行い,高い品質のドキュメントを作成することを目的とする.
