% !TEX root = ../zenkoku.tex
\section{はじめに}
ソフトウェア開発において,チーム内で熟練した開発者(以降,熟練者)はドキュメントを作成することがある.\cite{bib:ozawa}

% FIXME: なんか構造がおかしい,どのドキュメントに焦点を当てるかはもっと簡潔にかつ後に触れたほうが良さそう
% そもそも説明したいことに対して文量が多すぎるように思える
ソフトウェア開発におけるドキュメントには,会社等の組織の中で開発されるシステムに対するドキュメント,OSS (Open Source Software) 開発プロジェクトで開発されるシステムのドキュメント,ある組織が別の組織・個人に対して公開する開発者向けSDK(Software Development Kit)に対するドキュメントなど複数の状況・用途が考えられるが,今回は開発組織における熟練者が社内の他の開発者向に対し知識の共有を目的として作成するドキュメントに焦点を当てる.

情報工学においてはドキュメントを作成しなければいけない状況が存在するが,ドキュメントを執筆することで開発者が実装等に使えた時間は減ることになってしまう.
またドキュメントは充足したという状況を測ることが難しい,ドキュメント化対象のシステムが多すぎる,などの問題も存在する.
知識の共有を目的としてドキュメントが作成されるのであれば,熟練者への属人性が高いサブシステムに対してドキュメントが無いものを洗い出すことでドキュメント化対象を絞ること,優先度を付けることが可能なのではないかと考えた.

% FIXME: 前の分との繋がりが不自然,もう少し丁寧な導入無いかな
% FIXME: 「知識の呪縛」という概念はここ以降で使うつもりがないので出来れば用語を出さずに概念だけ説明して,残りは引用元を参照してね,という形にしたい
教育学における「知識の呪縛」(すでに理解した情報を知らないものととして想定することは難しいこと)\cite{bib:kaneda}は情報工学においても起きていることなのではないかと考える.例えば,熟練者が執筆したドキュメントは他の開発者にとって十分な情報を満たしたものとならない(知識の呪縛による認知バイアスがかかるため)などが挙げられる.

% FIXME: この段落が前の段落から飛躍しすぎているように感じる
ドキュメントが良いものであれば,ドキュメントが対象とする人の理解度が上がると考えられる.そこで本稿では,ドキュメント作成前にドキュメント対象者の理解度を計測し,これを元にドキュメントを作成する.

ただ理解度は個人の認知に基づく指標であるため,本稿では教育心理学にて知られている改訂版ブルーム・タキソノミー\cite{bib:nakao}という枠組を使い,理解度の数値化を行う.
