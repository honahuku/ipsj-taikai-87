% !TEX root = ../zenkoku.tex
\section{理解度を計測する指標}
理解度を個人の認知だけではなく,指標を元に多角的に評価するための枠組みは複数あるが,そのうちの1つであり教育学にて用いられることのある「改訂版ブルーム・タキソノミー」\cite{bib:anderson}に注目することにする.

ブルーム・タキソノミーでは,学習者の行動を認知的領域,情意的領域,精神運動的領域の3つに分類する.
このうち認知的領域は,記憶,理解,応用,分析,評価,創造,の6段階に分けられる.

