% !TEX root = ../zenkoku.tex
\section{アンケートの内容}
アンケートの内容は以下の通りである

\vspace{0.5cm}

このサービスについてこれまでの質問を踏まえあなたの理解を教えて下さい

\textbf{(注釈)}
\begin{itemize}
    \item 1~6のうち6に近いほど理想の状態だと考えてください
    \item 「認識・記憶できる」のうち例に挙げられている項目のすべてを満たさなくても「認識・記憶できる」に該当する場合はありえます。下記文章はあくまで参考程度に考えてください
  \end{itemize}
(理解度を選択するための具体例)

\renewcommand{\arraystretch}{1.5} % デフォルトの1.5倍にする
\scriptsize
\begin{tabular}{|p{1.3cm}|p{2.4cm}|p{2.9cm}|}
    \hline
    \textbf{レベル} & \textbf{説明} & \textbf{例} \\
    \hline
    1. 認識・記憶できる & ツール上の情報を認識し利用する & ・このツールで特定のCPU使用率のグラフを見つけられる \\
    \hline
    2. 理解できる & 情報を解釈し、説明や比較ができる(言い換える、例を上げる、要約する、比較する) & ・各モジュールが何を担当しているか説明できる \\
    \hline
    3. 応用できる & 学んだ知識を具体的な状況で使える(当てはめる、手順通りに実行する) & ・新しいパラメータをAPIに追加し、動作確認ができる \\
    \hline
    4. 分析できる & 情報を分解し、要素間の関係を理解できる(区別する、分類する、原因・理由を見つける) & ・エラー発生原因をログやコードベースから特定できる \\
    \hline
    5. 評価できる & 批判的に情報を評価し、結論を導き出せる(判断する、チェックする) & ・新しい要件を検討し、それが現在のアーキテクチャに適合するかを判断できる \\
    \hline
    6. 創造できる & 新しいアイデアや設計を作り出せる(仮説を立てる、計画する、デザインする、設計する) & ・大規模な変更を伴う新しい機能を提案し、それを設計・実装できる \\
    \hline
    \end{tabular}
\normalsize % 他の部分の文字サイズを元に戻す
