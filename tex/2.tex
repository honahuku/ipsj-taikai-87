% !TEX root = ../zenkoku.tex
\section{アンケートの内容}
アンケートの内容は以下の通りである\newline

% 脚注番号は数字で振る
\renewcommand{\thefootnote}{*\arabic{footnote}}
このサービスについてこれまでの質問を踏まえあなたの理解を教えて下さい\footnote{これより前にアンケートの回答を円滑にするためにシステムに対する設問を行っているが,本稿ではブルーム・タキソノミーを元に作成した設問のみ扱う}\footnote{アンケートの選択肢には「認識・記憶できる」の前の段階として「認識・記憶できない」を追加した}\newline

(注釈)
\begin{itemize}
    \item 1~6のうち6に近いほど理想の状態だと考えてください
    \item 「認識・記憶できる」のうち例に挙げられている項目のすべてを満たさなくても「認識・記憶できる」に該当する場合はありえます。下記文章はあくまで参考程度に考えてください
\end{itemize}
(理解度を選択するための具体例)

\renewcommand{\arraystretch}{1.5} % 表の隙間がデフォルトだと小さすぎるので標準の1.5倍にする
\scriptsize
% カウンタを作成
\newcounter{rownumber}

\begin{tabular}{|p{2.5cm}|p{4.2cm}|}
    \hline
    \textbf{レベル} & \textbf{説明と例} \\
    \hline
    \stepcounter{rownumber}\arabic{rownumber}. 認識・記憶できる & ツール上の情報を認識し利用する \newline 例:このツールで特定のCPU使用率のグラフを見つけられる \\
    \hline
    \stepcounter{rownumber}\arabic{rownumber}. 理解できる & 情報を解釈し、説明や比較ができる \newline 例:各モジュールが何を担当しているか説明できる \\
    \hline
    \stepcounter{rownumber}\arabic{rownumber}. 応用できる & 学んだ知識を具体的な状況で使える \newline 例:新しいパラメータをAPIに追加し、動作確認ができる \\
    \hline
    \stepcounter{rownumber}\arabic{rownumber}. 分析できる & 情報を分解し、要素間の関係を理解できる \newline 例:エラー発生原因をログやコードベースから特定できる \\
    \hline
    \stepcounter{rownumber}\arabic{rownumber}. 評価できる & 批判的に情報を評価し、結論を導き出せる \newline 例:新しい要件を検討し、それが現在のアーキテクチャに適合するかを判断できる \\
    \hline
    \stepcounter{rownumber}\arabic{rownumber}. 創造できる & 新しいアイデアや設計を作り出せる \newline 例:大規模な変更を伴う新しい機能を提案し、それを設計・実装できる \\
    \hline
\end{tabular}

\normalsize % 他の部分の文字サイズを元に戻す
