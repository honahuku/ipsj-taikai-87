% !TEX root = ../zenkoku.tex
\section{考察}
熟練者がチームに1人いる状況を想定し,結果から熟練者への依存度が高いようなシステムを洗い出すことにした.

熟練者の指標としては以下の視点を総合的に判断し,当てはまる上位1人を今回の考察の対象にした.
またこれらの指標はOSS (Open Source Software) 開発プロセスに関する研究\cite{bib:mockus}におけるコア開発者の指標を参考にした.

\begin{itemize}
	\item アンケートの対象とする各サブシステムに直近1年間最も多くデプロイした開発者
	\item 新機能の実装やコードベースの管理に多く取り組んだ開発者
	\item 開発タスクの起票から本番環境へのデプロイ速度が早い開発者
\end{itemize}

% TODO: ここに依存度の数値についての解釈を追加
熟練者への依存度として表示された数値としては

熟練者への依存度の高さに注目すると,システムH,A,B,Nなどにおいて熟練者に強く依存しているが,システムOやインフラ設問γのように差がほとんど見られないようなものもあった.
またシステムKやシステムHのバックエンド領域など熟練者,平均ともに理解度の低いものも見受けられ,これは開発組織内で誰も詳細を知らないシステムとなっていると考えられる.

この結果を開発組織内に対し討論する機会を設けたところ以下のような意見が出た.

% ここの書き方考え中
- 小規模なチームでは特定の人物へ依存しすぎるとチームメンバーの退職などで開発が立ち行かなくなることがありえる
- 熟練者への依存度を下げたい という思いはあるが何から始めたらいいかわからない状況がある. good First issue とかを用意してはどうか
- 出てきた理解度の平均,熟練者への依存度共に大きな認識の相違はなさそう

Looker の理解度の平均が高いが,これはアドプラのエンジニアがみんな広告関連で分析のために Looker を使うためであり,それがアンケートの結果にも現れている



属人性が低いシステムの特徴hoge
高いシステムの特徴hoge




