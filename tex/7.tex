% !TEX root = ../zenkoku.tex
\section{考察}
熟練者がチームに1人いる状況を想定し,結果から熟練者への依存度が高いようなシステムを洗い出す.

結果を見ると,理解度の差はシステムH,A,B,Nなどにおいて特に高いが,システムOやインフラ設問γのように差がほとんど見られないようなものもあった.
またシステムKやシステムHのバックエンド領域など熟練者,平均ともに理解度の低いものも見受けられる.

今回アンケートの対象としたサブシステムには,システムAやシステムNなど熟練者単独で実装したものがある.
また,システムBはドキュメントが不足しており,以前から実装難易度が高いと声が上がっていた.
これらは\ri{img:rikai}でも理解度の差が大きくなっている.
また,ツール2は開発者が頻繁に触れるものであり,これは理解度の差も小さい.

% ここから全般的な話
そのため,理解度の差が正の数かつ絶対値が大きいほど熟練者への依存度が高いといえる.
また,理解度の差がゼロに近ければ,熟練者への依存度は平均的で,あり,これはそのシステムを全員が理解しているかもしくは全員が理解していないどちらかである.
理解度の差が負の数かつ絶対値が大きい場合は,熟練者への依存度が低いといえる.
ただ,依存度が低いシステム全てが良いというわけではなく,システムEのようにシステムに対しての理解度が高い人がほとんど存在しない状況もある.
この場合は理解度の差だけでなく理解度の平均も低くなっている.

よって,熟練者への依存度の高さを改善したければ\ri{img:rikai}の理解度の差が大きいものを優先的に取り組むとよく,運用期間が長く理解度が高い人がほとんど存在しないものを改善したい場合は,理解度の差がゼロに近いようなものを検討すると良い.
