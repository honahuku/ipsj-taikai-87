% !TEX root = ../zenkoku.tex
\section{考察}
結果を見ると,熟練者への依存度の高さにおいてはシステムH,A,B,Nなどにおいて熟練者に強く依存しているが,システムOやインフラ設問γのように差がほとんど見られないようなものもあった.
またシステムKやシステムHのバックエンド領域など熟練者,平均ともに理解度の低いものも見受けられ,これは開発組織内で誰も詳細を知らないシステムとなっていると考えられる.

今回アンケートの対象としたサブシステムにおいては,システムAやシステムNなど熟練者単独で実装したもの,システムAやシステムBなどドキュメントが不足しているものなどが見受けられ,これらは依存度が高くなっている要因である.
また,ツール2のように開発者が頻繁に変更を加えるようなもの,システムEのように運用期間が長く理解度が高い人が少ないもの,などが依存度が低い要因である.
ただ,依存度が低ければ低いほど良いというわけではなく,システムEのような状況はシステムに対しての理解度が高い人が少ない状況を表している.

% TODO: ここから全般的な話
そのため,`熟練者への依存度の高さ` として算出した数値は,数値が正の数かつ絶対値が高いほど熟練者への依存度が高いといえる.
また,ゼロに近ければ熟練者の理解は平均的な理解度と近く,そのシステムを全員が理解しているかもしくは全員が理解していないどちらかである.
数値が負の数かつ絶対値が高い場合は熟練者への依存度が低いといえる.

よって,熟練者への依存度の高さを改善したければ\ri{img:rikai}の熟練者への依存度が大きいものを優先的に取り組むと良く,運用期間が長く理解度が高い人が少ないものを改善したい場合は理解度の平均と熟練者の理解度の両方がゼロに近いようなものを選択すると良い.
