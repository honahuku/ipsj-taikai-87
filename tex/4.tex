\section{アンケート対象者}
(さらりと流す程度)

属人性が高い(だれか詳しい人はいる)が,ドキュメントが無いものを洗い出し,その詳しい人にドキュメントを書いてもらう

ブルームタキソノミーについての知識がない開発者に答えてもらった



アンケートではシステムごとのブルーム・タキソノミーに基づく認識のみを回答させたわけではない.
最初からブルーム・タキソノミーを元に作成した6段階の認知過程の次元のいずれかを回答することは,本来教育目標の分類として作成されたブルーム・タキソノミーの抽象的な説明のみで情報工学におけるシステムの認知度を評価することになり,異なる分野の評価法を自らが担当する分野に適用するとも言え,困難と予想した.そのため,アンケート作成者がアンケートが想定する状況の具体例を説明及び事前の設問として加えた.

TODO: 設問の具体例を追記


以下にアンケート回答の流れを示す.

組織において開発・運用しているシステムに対する理解度の確認を行うアンケートを作成する
開発者がアンケートに回答する
ドキュメント執筆者に対してFBを行う
ドキュメントを執筆する

誰も知らないがドキュメントも無いものについては今回は対象外
ドキュメントは書いてもきりが無いので,優先度を付けて対応したい(背景)
ドキュメントを書くことにより開発する時間が減る(追記したい),が書かなければいけない(こっちは述べてる)
