% !TEX root = ../zenkoku.tex
\section{まとめ}
ブルーム・タキソノミーに基づいたアンケートによって理解度を数値として表し,どの
% WIP 既存の手法の問題
・アンケートによって執筆すべきとされるシステムを洗い出すことが出来た
・ただブルーム・タキソノミーの6分類を学習者に評価してもらうことによる同じ数値でも認識の相違がある
・これは hona が書いたので hona が認識出来ていない問題はわからない


% WIP:手法の良いところ
% これは新情報なのでまとめじゃなくて分析とかに書きたい
認識と相違ない数値を可視化できたのは意義がありそう


% WIP
今後の展望を書く?
- ドキュメントの足りてる具合を数値化
- 何がわからないかを自由記述方式のボックスとかを用意することで具体的なFB内容をヒアリング
- 定期的なアンケートとフィードバックによって改善を回す
- 知識の断絶を防ぐためにこの手法を用いることが出来るかも
- 理解度をFBしたことによってドキュメンテーションが改善されるかどうかを比較する
- 何らかの意味が現れていそうなデータが出て,それに意味付けをしたが実際にFBで改善されたのかという検証をしたい(この研究が最終的に到達しうる点)
