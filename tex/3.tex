% !TEX root = ../zenkoku.tex
\section{アンケートの実施と対象者}
本稿ではブルーム・タキソノミーにおける6段階の認知段階を情報工学に適用し,認知レベルを理解度を示す指標として用いて活用することを目指す.
適用の手法としてはブルーム・タキソノミーに基づいたアンケートを作成し,これをドキュメント利用者に回答してもらい各サブシステムに対する理解度を測定する.
これをドキュメンテーションに役立てることを目的とする.

アンケートの対象者は,山川と同じ開発組織に所属し広告配信システムの開発・運用に携わる9人の開発者となる.
回答はこの全員から得られた.
これらのアンケート回答者はブルーム・タキソノミーについての事前知識が与えられないままアンケートを回答している.
アンケート回答者はブルーム・タキソノミーを元に作成されたアンケートを用いて自分自身の認識を回答しており,回答者とは別の評価者等がアンケートを記入することは今回想定していない.
